\chapter{Introduction}

\section{Motivation}
Ontology Reuse
Despite promises to the contrary, ontologies are infrequently re-used. This results in a massive duplication of effort for any ontology.

Ontologies are everywhere and are spreading! For example, they are seeing widespread use in the digital humanities and increasingly across different sciences. Or under different names, such as knowledge graph schemas.

In the recent dagstuhl seminar on knowledge graphs, one of the grand challenges of the community was to define, demonstrate, and make available a methodology that emphasises FAIR data practices. Of particular note to this proposal are I- for interoperable and R for reusable. 

And of course, saving time during the ontology engineering phase would be welcome.

In order to address our problem, we want to come up with a methodology for creating ontologies that are easy to reuse. 

How to develop an ontology engineering methodology and its tools to make it easier for ontology engineers to re-use ontological artifacts?

Hypothesis: Modular ontology engineering with Ontology Design Patterns (ODP) makes reusing ontological artifacts easier.

More concretely, we posit that a modular ontology engineering paradigm and methodology based on the reuse and propagation of Ontology Design Patterns will in fact, make re-using ontological artifacts easier.

--- 
explain what odps are, consider talking about why using odps is a good idea, and maybe an example.

\section{Overview}
What is CORe, specifically.
How we intend CORe to be used.

\section{Organization}
namespaces

categories

documentation